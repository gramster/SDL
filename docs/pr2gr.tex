\docomentstyle[a4]{article}
\begin{document}
\title{Converting SDL/PR to SDL/GR}
\author{Graham Wheeler\\
Data Network Architectures Laboratory\\
Department of Computer Science\\
University of Cape Town}
\maketitle

\section{Introduction}

This document considers the problem of converting SDL/PR to
SDL/GR. The problem can be separated between two classes of
objects, essentially representing nodes (e.g. blocks, procedures,
etc) and arcs (e.g. signalroutes, channels, etc) of a directed
graph. For objects that represent nodes, the problems to be solved
are:

\begin{itemize}
\item where the node should be positioned;
\item the size of the node; and
\item the layout of the node's `children'.
\end{itemize}

For objects that represent arcs, the problem is one of identifying
the end points and junctions of $90^o$ turns, as well as placement
of the arc label.

\section{Remote References}

A problem with the graphical form of SDL is that diagrams may become
very cluttered. To prevent this, remote references may be used. To
simplify the automatic layout problem, we assume that all references
are remote. This allows the simplest and smallest representation of
processes, procedures, blocks, and so on.

\section{The System}

The system consists of a block labelled at top-left with the system
name. The block is divided into two parts; at the top are one or more
pages containing the textual form of the top level type and signal
definitions, and 
below this is a diagrammatic view of the constituent blocks,
connected with arcs representing the channels, the arcs being 
labelled with the signals that may pass through the channels
in each direction. The arcs may also join the perimeter of the
system block if the channels are connected to the environment.

\section{Blocks}

A block is much like a system except that the diagrammatic part now
shows the block's processes and their connections via signalroutes.

\section{Processes}

A process conists of a labelled block, with one or more pages
containing the declarations local to the process, and diagrammatic
trees representing the state transitions. If there is insufficient
space the description can be split into multiple pages.

Procedures are essentially the same.

\section{Basic Nodes}

Certain nodes cannot be decomposed any further. These nodes have as
a `child' a label consisting of the non-generic part of the phrase
representation. Such nodes include:

\begin{itemize}
\item nodes indicating the input states in an SDL state construct;
\item nodes indicating the next state, stop or return, at the end
of SDL transitions;
\item nodes indicating decisions, tasks, signal inputs and outputs,
procedures calls and process creations.
\end{itemize}

The size of the node can be computed from the size of the text label.
The position of the node is typically immediately below the preceding
node in a state sequence; exceptions to this are input nodes (as
there may be several input nodes following a state node), and nodes
following decision nodes.

In an SDL state paths can branch but not join. This is because the
joining of paths is represented using disconnected join and label 
nodes. A consequence of this is that a state sequence forms a tree
rooted in the input state node.

\section{Navigation through the System}

\end{document}
