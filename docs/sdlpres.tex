\documentstyle[a4,epsf,12pt]{article}
\begin{document}
\title{An SDL Toolset for Distributed System Development}
\author{Dr. Graham Wheeler\\Prof. Pieter Kritzinger\\
{\em Data Network Architectures Laboratory}\\
{\em Department of Computer Science}\\
{\em University of Cape Town}}
\maketitle

\begin{center}
{\bf Abstract}
\end{center}

We describe a new CASE toolset SDL*Design, 
which is based upon the International Telecommunication
Union's Specification and Description Language (SDL). We discuss the
need for such tools, and motivate our choice of SDL. The components
of the tool are described followed by a discussion of how it is used.
We conclude by mentioning further tools that are expected to
interwork with 

\section{The New Software Crisis}

The paradigms for software development are in a period of upheaval.
Several causes can be identified, amongst them:

\begin{itemize}
\item software complexity
is increasing to the point that traditional structured design approaches 
are becoming inadequate;
\item for similar reasons, there is an increasing demand for
reuseable components, both at the implementation and at the
specification and design levels;
\item the ubiquitous personal computer has accelerated the distribution
of computers over wide areas with the need to share services
and resources;
\item the increasing use of GUI environments. 
\end{itemize}

\section{Changing Paradigms for Software Engineering}

This upheaval has  resulted in the increased use of the following
computing paradigms:

\begin{itemize}
\item object-oriented analysis, design, implementation and testing;
\item client-server computing;
\item event-driven programming.
\end{itemize}

These paradigms are closely related; for example, event-driven systems
can be elegantly implemented using OOP, client-server systems are typically
event-driven, and GUIs are event-driven OO systems which may be
client-server based (e.g. X Window).

These types of systems cannot be designed using traditional top-down
methods. The top-down method often allowed the design stage to be 
largely bypassed, with design being a by-product of implementation.
A proper design stage in the software
engineering process is now much more critical. This is particularly
true in object-oriented systems where a badly designed class 
hierarchy may cripple the development process and will certainly
fail to provide reuse.

\section{Design in the Telecommunications Industry}

As a result of the demand, much recent research has focussed on
object-oriented analysis and design techniques, although at this
stage the results are still largely immature and experimental.
There is, however, one field in which event-driven OO
systems have long been known, namely telecommunications
engineering; the first object-oriented language
(Simula, 1967) was largely designed by Scandinavian telecommunication
engineers.
Since the early 1970s this community, through the auspices of the
International Telecommunication Union (ITU, formerly CCITT),
has developed a specification and
design notation called SDL (Specification and Description
Language)\cite{itusdl}.
Originally a graphical notation, an equivalent textual notation 
has also been introduced. The notation allows the specification of
the hierarchical structure
of a software system, as well as the internal dynamics of the processes
within the system, and their communication with each other and
the external environment.
The formal basis of the notation allows
for automated processing for the purposes of simulation, validation,
performance evaluation, and as a basis for implementation.
SDL has recently been enhanced with object-oriented data types,
but is otherwise a mature methodology in use by many of the world's 
largest telecommunication companies. For example, AT\&T developed
their 5ESS switch using SDL as the specification and design notation.
The SDL specification of the was formally validated by an automatic
validation program, the largest such validation yet
performed\cite{sdlholz}.

The SDL notation is based on a communicating finite-state machine model.
In addition to the notation for describing systems and processes,
it includes a notation for specifying message sequence charts specifying the
allowed sequences of message exchanges. 

The usefulness of the communicating FSM model and
graphical design notations has not
escaped other researchers in object-oriented analysis and design. Almost
all currently used OOA/OOD methodologies use graphical notations with
message-sequence charts (adopted from the telecommunication industry)
and timing diagrams (adopted from the computer hardware industry).
Furthermore, they all base the specification of dynamic behaviour on
communicating finite-state machines.
It is no surprise, then, that SDL has been used as the basis for
two object-oriented software engineering methodologies, 
Objectory\cite{objectory} and SISU\cite{sisu}.

\section{Objectory}

Objectory {Object FactOry} is a powerful OOA/OOD method developed by Ivar 
Jacobsen, which attempts to address the entire software 
life-cycle. It was developed from experience in building telephone exchange 
systems at Ericsson. Jacobsen was involved in the development of both 
Simula and SDL, and is one of the most experienced practioners in both
object-oriented techniques and the design of real-time distributed 
systems. The block structuring methods developed at Ericsson are the 
basis of both SDL and Objectory.
The dynamic semantics of Objectory are realised through a notation
heavily based upon SDL (although the methodology allows
other notations to be used if desired).

\section{SISU}

In 1989 a group of companies and research institutes in Norway initiated
the SISU project, having as its main goal to find and implement viable
solutions to the problem of how systems developers can gain control over
the development process, reduce costs and lead times, and improve product
quality. The result is a design-oriented
methodology that extends from requirements analysis through to
implementation (typically in C++). The design sepcification notation
used is SDL '92. The SISU group argue that SDL's potential for supporting
design-oriented systems engineering sets it apart from methodologies
based on other notations.

SISU project members included experienced object-oriented designers
and telecommunication engineers. The methodology has been applied
in industry, for example in the development of MAREK, a fully-
automatic telephone and telex systems for ships using INMARSAT.

\section{The SDL*Design Toolset}

For a design methodology to be really useful, tools are needed to
aid in the creation, maintenance, and if possible, analysis, of
specifications in the associated notation. Ideally such a tool
will do all of the above, as well as generate source code as the
basis of an implementation (ensuring conformance to the specification).

The authors have experience in the development of such a tool, using ISO
Estelle (Extended State Transition Language) as the specification 
notation. Estelle is a Pascal-based notation with extensions for
concurrent communicating finite-state machines. However, Estelle has
not been widely used by industry, and is inadequate as a design notation,
as it is too implementation-oriented.

For reasons which are clear from the earlier discussion, we
decided that SDL was a more appropriate choice, and have thus
implemented a set of tools for the checking, simulation and analysis
of SDL specifications. The toolset has been decomposed in a modular
fashion into a number of separate components.
We  aim to enable these components to interoperate with
other tools developed by the Data Network Archtectures 
research group.
 
The structure of the toolkit is shown in Figure~\ref{sdltool}. 
Modules are shown as rectangles, while the data exchanged is shown as
ovals. The modules which have been implemented at the time of
writing are shown as solid blocks. The toolkit has been implemented
in C++ and runs under MS-DOS with MS-Windows 3.1 as well
as under UNIX with X and Motif.

\begin{figure}\centering
\epsfxsize=15cm
\epsffile{sdltool.ps}
\caption{\label{sdltool} Components of the SDL*Design toolset.}
\end{figure}

The components are:

\begin{description}
\item [Graphical editor] This supports the graphical form of SDL.
Graphical diagrams can be easily drawn, edited, and annotated.
The drawings can be saved to disk and recalled. The editor also
supports the export of the SDL/GR system to SDL/PR, or to Xfig
figure files. The latter can then be converted to numerous other
formats, including Encapsulated PostScript.
\item [Preprocessor] This performs macro expansion on a
specification, removes comments, handles conditional compilation,
and combines specifications that are split across multiple files 
(there is currently no support for separate or incremental
compilation);
\item [Syntax checker/AST generator] This checks the syntax of the
specification, and builds an abstract syntax tree representation
based on the abstract syntax in the SDL standard;
\item [Petri Net model generator] This module is currently under 
development. Its task is to translate systems that have been compiled
into AST form in Petri net models that can be validated and
have their performance analysed using Petri net tools being
developed by the Data Network Laboratory in conjunction with the
University of Dortmund.
\item [Semantic checker/VM code generator] This performs semantic
checking of the abstract syntax tree, and outputs interpretable code
targeted at a set of virtual machines specially designed for SDL;
\item [Interpreter/debugger] - This creates and manages the SDL
virtual machines (which represent the system, blocks, processes and
procedures of the specification). It has a simple text-based user
interface but allows sophisticated event-tracing, fine control over
execution, and access to symbolic debugging information;
\item [Graphical simulator] This is a front-end to the interpreter
which will provide a graphical interface to the interpreter,
and will ultimately include a visual representation of the dynamics of system
execution;
\item [Performance analyser] This is another front-end to the interpreter,
which will produce event traces which will subsequently be analysed and from
which semi-Markov
models of the performance behaviour of the system will be derived;
\item [Model solver] This solves models generated by the performance
analyser and allows these models to be used predictively;
\item [Validator] This will perform validation and possibly performance 
analysis via state-space exploration of the system;
\item [Decompiler] This produces pretty-printed textual SDL from
an abstract syntax tree; it can be used recreate specifications that
are accidentally deleted as well as for automatic formatting;
\item [Disassembler] This produces a disassembled listing of the SDL
virtual machine program code;
\item [C++ code generator] This module converts the compiled system
into a set of C++ classes. Each data type, signal type and process type
is converted to an appropriate class definition. The resulting code
can be compiled and linked with a run-time support file to produce
an executable simulator which executes hundreds of times faster than
the interpreter. By modifying the run-time support file, the C++
classes can be used as the basis for an implementation generated
directly from the design specification.
\end{description}

\section{Using the Toolkit}

At present, the toolset has the following uses:

\begin{itemize}
\item Specifications can be entered and edited graphically
using the graphical editor, or can be entered in textual 
form using any text editor.
\item The specifications can be checked for syntactic and semantic
errors.
\item The performance of the system can be evaluated by executing
the compiled specification, or by generating and executing a C++ 
version of the specification.
\item Output from the executions can be used as data to be analysed 
to create formal performance models of the system.
\item Further validation is possible by
including assertions within the specification.
\end{itemize}

\section{Conclusion}

The toolkit currently supports a large subset of SDL '88.
Most of the omitted features are structural devices used for specifying
a system at a variety of levels of detail. As our use of SDL
is oriented towards executions, we do not need these mechanisms.
Were they needed, similar effects can be obtained using
conditional compilation.

The toolkit must still be updated to support the SDL '92 extensions.
The main extensions are the introduction of parameterised types
(similar to C++ templates) and inheritance of data types. SDL's
data type facility supports abstract data types in which the
properties of operations on a type are expressed axiomatically,
rather than the operations being described algorithmically;
our requirement that these be made concrete for the purposes of
implementation has led us to simplify our implementation 
of data types from the SDL '88 standard (other SDL researchers
have used ASN.1 or similar methods for defining concrete data;
we believe this is unnecessarily complex).
SDL '92 goes some
way to addressing this implementation problem.

\begin{thebibliography}{9}
\bibitem{itusdl} International Telecommunication Union {\em Recommendation
Z.100: Functional Specification and Description Language (SDL)} 
International Telecommunication Union, 1993.
\bibitem{objectory} Ivar Jacobson et. al. {\em Object-Oriented
Software Engineering} ACM Press, 1992.
\bibitem{sisu} Rolv Braek and Oystein Haugen {\em Engineering Real
Time Systems} Prentice-Hall, 1993.
\bibitem{sdlholz} Gerard Holzmann {\em Practical methods for the 
formal validation of SDL specifications} Computer Communications V15N2,
March 1992.
\end{thebibliography}
\end{document}



