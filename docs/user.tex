\documentstyle[a4,12pt]{article}
\begin{document}
\author{Graham Wheeler}
\title{SDL Toolset User's Guide}
\maketitle

\section{Introduction}

This is a short guide to the SDL*Design Toolset v1.0.
{\em Note that this is an alpha release of the toolset!}
The programs in the toolset are:

\begin{itemize}
\item {\tt sdlcc} -- the multi-pass driver;
\item {\tt sdlpp} -- the preprocessor;
\item {\tt sdlc1} -- the first pass of the compiler (parser and AST
constructor);
\item {\tt sdlc2} -- the second pass of the compiler (checker and code
generator);
\item {\tt sdlrun} -- the interpreter/debugger.
\item {\tt sdldis} -- the disassembler;
\item {\tt sdldc} -- the AST decompiler;
\end{itemize}

The toolset supports a large subset of ITU SDL-88/PR.

Each is now described in turn.

\section{The Multi-Pass Driver {\tt sdlcc}}

This is the command usually used to invoke the three phases of
compilation, namely preprocessing (done by {\tt sdlpp}),
parsing and syntax checking (done by {\tt sdlc1}), and semantic 
checking and code generation (done by {\tt sdlc2}).  {\tt sdlcc}
invokes each of these in turn, provided the preceding phase
ran with no errors.

The file containing the specification must be given in the command
line. If the file cannot be opened, the compiler will try to open it
with a `{\tt .sdl}' extension. 

If remote definitions are used in the specification, these can
be given in separate files. In this case, all files must be given on
the command line, with the file containing the main system
specification first.

{\tt sdlcc} passes any file arguments on to the preprocessor.
Any other arguments are either ignored or passed on to one of
the three passes. The latter occurs when an argument is given 
which is of the form ``{\tt -P}{\em string}'' (in which case 
``{\tt -}{\em string}'' is passed to the preprocessor), 
``{\tt -1}{\em string}'' (parser) or ``{\tt -2}{\em string}''
(checker/code generator). For example, to pass the argument
``{\tt -H2}'' to the parser, invoke {\tt sdlcc} with a ``{\tt -1H2}''
argument.

If the last message produced by {\tt sdlcc} is ``Compilation
successful'', you can execute the interpreter {\tt sdlrun}.

Each of the three phases supports a {\tt -H} argument which
results in statistics about the heap use being printed. The
{\tt -H} can be followed by a `level' value indicating the amount of detail
required (maximum detail at level 3 or more).

\section{The Preprocessor {\tt sdlpp}}

The preprocessor is intended as a macroprocessor for SDL. At the
time of writing macros are not yet supported. Nontheless, the
preprocessor coalesces the input files, removes comments, and adds
its own special comments identifying the correspondence between
lines in the coalesced file and lines in the original input file(s).
This allows errors reported by the other phases to refer to the
unpreprocessed source.

Apart from the {\tt -H} argument, the preprocessor can take a
``{\tt -o} $<${\em file}$>$'' argument specifying the file to
use for output. If this is not specified the output is sent to the
standard output.

\section{The First Pass {\tt sdlc1}}

The first pass of the compiler parses the specification and
builds an abstract syntax tree (AST) representation, which is
written to the file {\tt ast.out}. Virtually no checking other than
syntactic checking is done.

The file containing the specification must be given in the command
line. If the file cannot be opened, the compiler will try to open it
with a `{\tt .sdl}' extension. In fact invocation of this phase is
very similar to that of the preprocessor - in the case where a 
specification contains no macros {\tt sdlc1} can be invoked directly
without any preprocessing being necessary.

The following SDL constructs are {\em not} currently supported:

\begin{itemize}
\item string types; the toolset supports characters,
booleans, naturals and integers, reals, times and durations,
and arrays and structures of these, as well as enumerations.
\item service decompositions;
\item block and channel substructures;
\item signal refinements;
\end{itemize}

The following have not been implemented at the time of writing
but will be real soon:

\begin{itemize}
\item transition alternatives;
\item macros.
\end{itemize}

Also, parameters passed through nested procedures, and the predefined
variables ({\tt sender}, {\tt self}, etc) are untested at the time
of writing.
The first pass can take a {\tt -S} argument which just suppresses
the printing of the input file name. This is used when invoked
from {\tt sdlcc} as the input file name in that case is a that 
of a temporary file created as a `pipe' between the preprocessor
and the first pass.

\section{The Second Pass {\tt sdlc2}}

The second pass reads the {\tt ast.out} file produced by the first
pass, allocates addresses and indices to the various objects 
(such as variables, states, signals, parameters, etc), performs 
semantic checking on the tree, and, if all has gone well, generates
code for the system and links the code. The resulting executable
(well, interpretable) code is written to the file {\tt sdl.cod}. The
modified AST (with addresses/indices) is written to the file {\tt
ast.sym} for use by the interpreter/debugger.
 
The second pass can take an optional {\tt -v} (verbose) argument, in
which case it produces all sorts of arcane output. Also, a {\tt -l}
argument can be used to suppress linking (only for {\tt sdldis} 
hacking!).

\section{The Decompiler {\tt sdldc}}

The decompiler reads the {\tt ast.out} file produced by the first
pass and prints the tree in SDL. The result should be equivalent to
the original source, although the ordering may be slightly different,
and remote references are resolved (this is done by the first pass).
The decompiler can be used as a test to see that the compiler
parsed a specification correctly, and as a pretty printer.

\section{The Disassembler {\tt sdldis}}

The disassembler prints the contents of the code file {\tt sdl.cod}
produced by the second pass. It is not of much use except to those
who understand the underlying pseudocode!

\section{The Interpreter/Debugger {\tt sdlrun}}

If the first and second passes have been executed successfully (i.e.
no errors reported) then the interpreter can be used. At the time of
writing the interpreter does not yet support the external 
environment of the system. Also, support for procedures is
incomplete (although simple procedures can be used).

The interpreter command set is intentionally `obscure'. The idea
is to provide a simple but powerful interface which will later be
driven from a GUI or other user-friendly front-end. Thus, the
interpreter is not really intended to be executed by mere mortals.
However, it turns out that once you get the hang of it, it's actually
quite easy!

The interpreter can take a {\tt -L} command-line argument, in which 
case its output is sent to a file {\tt sdlrun.log} as well as to the
standard output. It also can take a {\tt -H} argument for heap 
statistics.

When the interpreter starts up, you should see some messages
something like this:

\begin{verbatim}
     0      C 1  
     0      C 2  
     0      C 3  
     0      D 2  
     0      C 4  
     0      C 5  
     0      D 4  
     0      C 6  
     0      C 7  
     0      D 6  
     0      D 1  
\end{verbatim}

These are {\em trace messages}. The interpreter executes the
system, requesting commands on occasion (exactly when will vary, 
as you will see), and produces trace messages as output. The
trace messages belong to several different classes. Each message
starts with the {\em time} of the event the message is reporting,
followed by a single character {\em class code}. The message classes
and remaining fields are described in the table below:

\small
\begin{tabular}{lclllll}\\
\hline
Event Type & Code & Fields &&&&\\
\hline
\hline
Process created & {\tt C} & {\em process} &&&&\\
Process destroyed & {\tt D} & {\em process} &&&&\\
\hline
Signal output & {\tt O} & {\em sender} & {\em signal} & {\em receiver} &&\\
Signal arrived & {\tt A} & {\em receiver} & {\em signal} &&&\\
Signal input & {\tt I} & {\em receiver} & {\em signal}& &&\\
Signal lost & {\tt U} & {\em receiver} & {\em signal} & {\em channel} &&\\
Implicit transition fired & {\tt T} & {\em receiver} & {\em signal} &&&\\
\hline
Transition fired & {\tt F} & {\em process} & {\em transition} &&&\\
State change & {\tt S} & {\em process} & {\em fstate} & {\em tstate} & {\em
fname} & {\em tname}\\
Source line change & {\tt L} & {\em file} & {\em line} &&&\\
S-code instruction & {\tt X} & {\em stack top} & {\em instruction} &&& \\
\hline\\
\end{tabular}
\normalsize

This is all very complicated, but can be made easier if you 
enter {\tt v}. This makes the traces verbose, showing process
type names and locations in the AST, and is easier to read to begin
with (entering {\tt v} again goes back to terse mode).
There is a good reason for the output usually being terse,
though - filters, which we describe later.

The messages at start-up are informing us of the 
creation of system and block processes, which then create initial 
processes, and then terminate. These are the only types of traces
that are enabled by default. To change the types of traces, you use
the {\tt m} command. This is followed by a set of letters
representing the trace class (these are the same as shown in the
trace messages except that process creation/deletion is selected
with the single character {\tt p}).
To disable one or more trace classes, precede the command with a minus.
For example, the following two commands disable the production of
process create/delete messages, and enable the production of output
and transition firing messages:

\begin{verbatim}
    -mp
    mof
\end{verbatim}

Enter {\tt ?m} to list the currently enabled trace classes.

Once you've selected the trace message classes you want to see,
you can start running the system. The interpreter holds a
table of active processes. Some of these should be active
already. To see the list of processes, enter `{\tt ?p}'.
The result will be something like:

\begin{verbatim}
     P3  
     P5  
     P7  
\end{verbatim}

(more if you're in verbose mode). These are the three active
processes, with process IDs 3, 5, and 7. For more information about
a process, enter `{\tt ?p}' followed by the process ID. You'll see
something like:

\begin{verbatim}
     P7           idle (0 )
     Port: p_data_req(1) [11]
     sn         = 0
     rn         = 0
     seq        = UNDEFINED
     tm         = 10
     T1   2        2        3       
     T2   1        4        4       
     T3   0        -        -
\end{verbatim}

This tells us that process 7 is in state {\tt idle} (which has index
0) and has three transitions. Transition T1 has fired twice, first 
at time 2 and most recently at time 3; transition T2 has fired once
at time 4, and transition T3 hasn't fired yet. There is a single 
signal queued for input, namely a p\_data\_req with argument 1. The
arrival time of the signal is time 11. Between the {\tt Port} line
and the transitions appear the names and values of variables and
parameters defined in the process.

To see the body of a transition, enter {\tt ?t} followed by the
process and transition indices. For example, to view the last
transition of the process in the example above we would use
{\tt ?t 7 3}.


To execute, enter `{\tt x}' followed by an optional
numeric count and then an {\em execution unit}. The interpreter
will continue execution until the number of the `events' specified
by the execution unit have occurred (the default being one), and
then ask for another command. This is why it was stated earlier that 
it varies at what point the interpreter asks for a command. 
The unit can be any one of {\tt I} (scheduler iterations), 
{\tt T} (time units), {\tt C} (scheduler choices - not implemented yet),
{\tt M} (trace messages), {\tt F} (transition firings),
{\tt L} (source line number changes), or {\tt X} (s-code instruction
executed). {\tt L} doesn't make much sense except for small numbers
as execution tends to jump all over the place with respect to the
original source.

For example, {\tt XF} executes until the next transition has fired,
{\tt X3F} executes three transitions before asking for another
command, and so on. You can combine using trace classes and the
{\tt M} execution unit to create simple breakpoints on signal
passing. For example, if you enable only signal output traces, then 
{\tt X5M} executes until five signals have been output.
Unfortunately, this isn't very specific - we may want to constrain
which process' outputs we are interested in, or which signals we are
interested in, for example. This is where filters are used.

Internally, every time a traceable event occurs, the {\em scheduler}
or currently executing {\em process} informs the {\em trace manager}.
The trace manager checks if the trace should be output, and if so,
tells the scheduler to do so. It makes its decision based firstly on
which trace classes are enabled, and secondly on a set of {\em
accept} and {\em reject filters}. These filters are simple patterns
matched against the fields in the trace message. A candidate message
is first checked against the accept filters; if it matches one it is 
output. If not it is checked against the reject filters; if it
matches one it is {\em not} output. If it doesn't match any reject
filter, it is output by default. Thus the reject filters override the
default, while the accept filters override the reject filters. The
reject filters narrow down the trace; the accept filters let selected
things through.

To set an accept filter, enter `{\tt A}' followed by an optional
count and the filter specification. 
To set a reject filter, enter `{\tt R}' followed by an optional
count and the filter specification. Deleting a filter is done 
by entering the command preceded by a minus (no count is allowed).

We'll get to the count shortly. The specification consists of a
letter identifying the trace class that we are filtering (these are
the same as for the {\tt M} command), an optional process ID (which
can be {\tt *} to match any), and up to three optional fields (each
of which can be {\tt *} to match anything). The three-field
limitation is why the terse output form is usually best when using
filters.

For example, the following commands filter out any transition firing
trace messages from process 3, and all signal output trace messages
of the signal with index 2:

\begin{verbatim}
     r f 3
     r o * * [2]
\end{verbatim}

The counts have different meanings for each type of filter. For a
reject filter, the count is the lifetime of the filter. After it has
caused the rejection of that many trace messages it becomes inactive
and will be reallocated by the scheduler at some point.
For an accept filter, on the other hand, the count represents a
`sleep' period - only {\em after} that many matches have been made
by the filter will it actually cause matching messages to be
accepted.

A couple of examples will help illustrate this:

\begin{verbatim}
     -m
     mf
     r10f
     xm
\end{verbatim}

This enables transition firing traces only, creates a filter that
rejects the next ten of these, and runs until the first trace message
output after the reject filter expires; i.e. it runs until ten or
eleven transitions have fired (the exact number depends on the last
execution unit).

\begin{verbatim}
     -m
     mo
     ro
     a10o 5
     xm
\end{verbatim}

This runs until process 5 has output 10 signals.

Judicious use of the interpreter command set can give you 
very powerful breakpoint functionality. This could be considerably
enhanced by providing full regular expression matching in filters,
but the performance cost may be too high.

At present, signals that are transported between blocks are delayed
for one time unit before arriving at their destination (within blocks
there is no delay). This whole mechanism will be upgraded, but the 
interpreter does offer a bit more for now. The {\tt C} command
offers control over the known channels (at the moment the interpreter 
only learns of each channel once an output occurs on it; a minor drawback).
{\tt ?C} lists the known channels. The last two fields are the
(single-valued at present) delay associated with the channel, and the
reliability (in percent). Initially these should be 1 and 100
respectively. You can change them be entering {\tt C} followed by
the channel table entry number, the new delay, and the new
reliability. 

Finally, if the system deadlocks, this will be reported, That is a
good time to enter `{\tt D}', which enables a trace of transition 
guard evaluation, and then `{\tt XF}' to force a reevaluation. If
you are logging to {\tt sdlrun.log} you can get a good idea of why
no transitions are enabled from the resulting output (it's too much
to deal with on the screen). List all the processes too, with the
{\tt L} command. So far I've gone through several iterations of 
the AB protocol spec deadlocking, and I've generally suspected the
interpreter (very new code), but it's always been the AB spec.
It's now quite easy to see what's up. The trace messages produced 
with the {\tt D} option on are not very compatible with filters,
so my advice is don't use them at the same time.

\newpage
\section{Interpreter Quick Reference}

\subsection{Trace Message Formats}

\begin{tabular}{ll}
{\em time} {\tt F} {\em pid} {\em tnum} & Transition fired\\
{\em time} {\tt T} {\em pid} {\em signal} & Implicit transition\\
{\em time} {\tt I} {\em pid} {\em signal} & Signal input\\
{\em time} {\tt U} {\em pid} {\em chan} {\em signal} & Signal lost\\
{\em time} {\tt O} {\em srcpid} {\em signal} {\em destpid} & Signal output\\
{\em time} {\tt L} {\em file} {\em line} & Source line change\\
{\em time} {\tt C} {\em pid} & Process created\\
{\em time} {\tt D} {\em pid} & Process destroyed\\
{\em time} {\tt S} {\em pid} {\em from} {\em to} {\em fname} {\em tname}
& State changed\\
{\em time} {\tt X [} {\em stack} {\tt ]} {\em instruction} & S-code
instruction\\
{\em time} {\tt A} {\em pid} {\em signal} & Signal arrival\\
\end{tabular}\\

In the above, {\em signal} contains the signal name, the first couple
of arguments, and the arrival time of the signal (this is in the
future in output trace messages). {\em pid}s are simply integers
in terse mode, but include the AST location and process type name in
verbose mode.

\subsection{Trace Classes}

\begin{tabular}{ll}
{\tt P} & Process creation and deletion (used in commands)\\
{\tt C} & Process creation (used in trace messages)\\
{\tt D} & Process deletion (used in trace messages)\\
{\tt O} & Signal output\\
{\tt A} & Signal `arrival' on input port queue\\
{\tt I} & Signal input\\
{\tt F} & Transition firing\\
{\tt L} & Source line change\\
{\tt U} & Signal lost\\
{\tt X} & S-code instruction executed\\
{\tt S} & State changes\\
{\tt T} & Implicit transitions\\
\end{tabular}\\

These are used in the {\em classlist}s in the trace control commands,
and in the {\em tc}s in filter commands. Filters for {\tt S} trace
messages are not supported.

\subsection{Trace Control}

\begin{tabular}{ll}
{\tt ?m} & Shows which trace classes are enabled\\
{\tt m} & Enable all trace classes except {\tt U}\\
{\tt m} {\em classlist} & Enable the specified trace classes\\
{\tt -m} & Disable all the specified trace classes\\
{\tt -m} {\em classlist} & Disable the specified trace classes\\
\end{tabular}

\subsection{Filters}

\begin{tabular}{ll}
{\tt ?a} & Lists accept filters\\
{\tt a} {\em cnt} {\em tc} [ {\em PId} [ {\em flds} ] ] & Set accept filter\\
{\tt -a} {\em tc} [ {\em PId} [ {\em flds} ] ] & Delete accept filter\\
{\tt ?r} & Lists reject filters\\
{\tt r} {\em cnt} {\em tc} [ {\em PId} [ {\em flds} ] ] & Set reject filter\\
{\tt -r} {\em tc} [ {\em PId} [ {\em flds} ] ] & Delete reject filter\\
\end{tabular}\\

{\em PId} and {\em flds} can be {\tt *} to match any, or must
match the corresponding whitespace-separated field in the trace
message exactly.

\subsection{Execution}

\begin{tabular}{ll}
{\tt x} {\em cnt} {\tt f} & Execute until {\em cnt} transitions fire\\
{\tt x} {\em cnt} {\tt i} & Execute for {\em cnt} scheduler iterations\\
{\tt x} {\em cnt} {\tt t} & Execute for {\em cnt} time units\\
{\tt x} {\em cnt} {\tt m} & Execute until {\em cnt} traces are accepted\\
{\tt x} {\em cnt} {\tt l} & Execute for {\em cnt} line number changes\\
{\tt x} {\em cnt} {\tt x} & Execute for {\em cnt} s-code instructions\\
{\tt x} {\em cnt} & Execute until {\em cnt} events of last class\\
{\tt x} & Repeat last execute command\\
{\tt Enter} & Repeat last execute command\\
\end{tabular}

\subsection{Channel Control}

\begin{tabular}{ll}
{\tt ?c} & Lists the known channels\\
{\tt c} {\em num} {\em delay} {\em reliability} & Changes channel attributes\\
\end{tabular}

\subsection{Other Commands}

\begin{tabular}{ll}
{\tt ?} & Shows help\\
{\tt ??}{\em class} & Shows help on a class of commands\\
{\tt ?p} & List process IDs of active processes\\
{\tt ?p} {\em pid} & Shows information about a specific process\\
{\tt ?t} {\em pid} {\em tnum}& Shows body of a specific transition\\
{\tt v} & Toggles verbose process identifiers on/off\\
{\tt d} & Toggles debug trace of transition guard evaluation on/off\\
{\tt ESC} & Interrupts execution (DOS only at present)\\
{\tt q} & Exits the interpreter\\
\end{tabular}

\newpage
\section{Sample Specification}

\begin{verbatim}
/* Simple Alternating-Bit example with loopback provider */

system ab_example;

        newtype Packet          /* Dummy data packet definition */
                array (0) of character;
        endnewtype

        signal
                u_data_req(Packet), u_data_rsp, u_data_ind(Packet),
                p_data_req(Integer,Packet), p_data_ind(Integer,Packet),
                p_ack_req(Integer), p_ack_ind(Integer);

        channel upper_Ch
                from User to Protocol
                        with u_data_req;
                from Protocol to User
                        with u_data_ind, u_data_rsp;
        endchannel upper_Ch;

        channel lower_Ch
                from Protocol to Provider
                        with p_data_req, p_ack_req;
                from Provider to Protocol
                        with p_data_ind, p_ack_ind;
        endchannel lower_Ch;

        /* Provider Block */

        block Provider;
                process ProvControl referenced;
                signalroute Comm
                        from ProvControl to env
                                with p_data_ind, p_ack_ind;
                        from env to ProvControl
                                with p_data_req, p_ack_req;
                connect lower_Ch and Comm;
        endblock Provider;
\end{verbatim}
\newpage
\begin{verbatim}
        block Protocol;
                process ProtControl referenced;
                signalroute PComm
                        from ProtControl to env
                                with p_data_req, p_ack_req;
                        from env to ProtControl
                                with p_data_ind, p_ack_ind;
                signalroute UComm
                        from env to ProtControl
                                with u_data_req;
                        from ProtControl to env
                                with u_data_ind, u_data_rsp;
                connect lower_Ch and PComm;
                connect upper_Ch and UComm;
        endblock Protocol;

        block User;
                process UserControl referenced;
                signalroute Comm
                        from UserControl to env
                                with u_data_req;
                        from env to UserControl
                                with u_data_ind, u_data_rsp;
                connect upper_Ch and Comm;
        endblock User;
endsystem ab_example;

process system ab_example/block Provider ProvControl(1,1);
        dcl seq Integer;
        dcl data Packet;
        start;
                nextstate idle;
        state idle;
                input p_data_req(seq,data);
                        output p_data_ind(seq,data);
                        nextstate -;
                input p_ack_req(seq);
                        output p_ack_ind(seq);
                        nextstate -;
endprocess;
\end{verbatim}
\newpage
\begin{verbatim}
process system ab_example/block Protocol ProtControl(1,1);
        dcl sn Integer;
        dcl rn Integer;
        dcl seq Integer;
        dcl data Packet;
        dcl tm Duration := 10;
        timer Timeout;
        start;
                task sn := 0;
                task rn := 0;
                nextstate idle;
        state idle;
                input u_data_req(data);
                        output p_data_req(sn, data);
                        set (now + tm, Timeout); 
                        nextstate wait;
                input p_ack_ind(seq); /* Do nothing */
                        nextstate -;
        state wait;
                input p_ack_ind(seq);
                        decision seq = sn;
                        ( false ) : /* do nothing */
                                nextstate -;
                        ( true )  :
                                task sn := (sn + 1) mod 2;
                                output u_data_rsp;
                                nextstate idle;
                        enddecision;
                input Timeout;
                        output p_data_req(sn, data);
                        set (now + tm, Timeout); 
                        nextstate -;
        state *;
                input p_data_ind(seq, data);
                        decision seq = rn;
                        ( false ) : /* do nothing */
                        ( true )  :
                                output u_data_ind(data);
                                output p_ack_req(rn);
                                task rn := (rn + 1) mod 2;
                        enddecision;
                        nextstate -;
endprocess;
\end{verbatim}
\newpage
\begin{verbatim}
process system ab_example/block User UserControl(1,1);
        dcl data Packet;
        dcl delayWait duration := 3;
        timer sendDelay;
        start;
                nextstate sendfirst;
        state sendfirst;
                provided True;
                        output u_data_req(data);
                        nextstate idle;
        state idle;
                input u_data_rsp;
                        set (now + delayWait, sendDelay);
                        nextstate -;
                input u_data_ind(data); /* consume */
                        nextstate -;
                input sendDelay;
                        output u_data_req(data);
                        nextstate -;
endprocess;
\end{verbatim}

\end{document}
